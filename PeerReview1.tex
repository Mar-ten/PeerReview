\documentclass[12pt]{article}
\usepackage{amsmath}
\usepackage{amsfonts}
\usepackage{mathrsfs}
\usepackage{lscape}
\usepackage{listings}
\usepackage{graphicx} % Allows for importing of figures
\usepackage{color} % Allows for fonts to be colored
\usepackage{comment} % Allows for comments to be made
\usepackage{accents} % Allows for accents to be made above and below text
%\usepackage{undertilde} % Allows for under tildes to take place for vectors and tensors
\usepackage[table]{xcolor}
\usepackage{array,ragged2e}
\usepackage{hyperref}
\usepackage{framed} % Allows boxes to encase equations and such
\usepackage{subcaption} % Allows for figures to be side-by-side
\usepackage{float} % Allows for images to not float in the document
\usepackage{booktabs}
%\usepackage[margin=0.75in]{geometry}
\usepackage[final]{pdfpages}
\usepackage{enumitem}
\usepackage[section]{placeins}

%%%%%%%%%%%%%%%%%%%%%%%%%  Function used to generate vectors and tensors %%%%%%%%%
\usepackage{stackengine}
\stackMath
\newcommand\tensor[2][1]{%
	\def\useanchorwidth{T}%
	\ifnum#1>1%
	\stackunder[0pt]{\tensor[\numexpr#1-1\relax]{#2}}{\scriptscriptstyle \sim}%
	\else%
	\stackunder[1pt]{#2}{\scriptscriptstyle \sim}%
	\fi%
}
%%%%%%%%%%%%%%%%%%%

\definecolor{mygrey}{rgb}{0.97,0.98,0.99}
\definecolor{codeblue}{rgb}{.2,0,1}
\definecolor{codered}{rgb}{1,0,0}
\definecolor{codegreen}{rgb}{0.3,0.33,0.12}
\definecolor{codegray}{rgb}{0.5,0.5,0.5}
\definecolor{codepurple}{rgb}{0.55,0.0,0.55}
\definecolor{codecyan}{rgb}{0.0,.4,.4}

\lstdefinestyle{mystyle}{
	backgroundcolor=\color{mygrey},   
	commentstyle=\color{codegreen},
	keywordstyle=\color{codeblue},
	stringstyle=\color{codepurple},
	numberstyle=\tiny\color{codegray},
	basicstyle=\footnotesize,
	breakatwhitespace=false,         
	breaklines=true,                 
	captionpos=b,                    
	keepspaces=true, 
	numbers=left,                    
	numbersep=5pt,                  
	showspaces=false,                
	showstringspaces=false,
	showtabs=false,                  
	tabsize=2
}
\lstset{style=mystyle}

\lstset{language=Matlab,backgroundcolor=\color{mygrey}}
\usepackage{lastpage}
\usepackage{fancyhdr}
\pagestyle{fancy}
%\lhead{\large{Nik Benko, John Callaway, Nick Dorsett, Martin Raming}} 
%\chead{\large{\textbf{ME EN 6960: Lab 1}}}
%\rhead{\today}
\cfoot{[\thepage\ of \pageref{LastPage}]}
\fancyheadoffset{.5cm}
\setlength{\parindent}{0cm}
\usepackage[left=.5in, right=0.50in, top=1.00in,bottom=1.00in]{geometry}
\usepackage{microtype} 
\usepackage{setspace}
\doublespace
%%%%%%%%%%%%%%%%%%%%%%%%%%%%%%%%%%%%%%%%%%%%%%%%%%%%%%%%%%%%%%%%%%%%%%%%%%


\begin{document}
\title{ Review of manuscript ``Determining the Laminate Orientation Code from a List of Possible Layups and Know Material via Strain Gauge Measurements" \\ \normalsize{ME EN 6960}}
\maketitle


\section{Synopsis:} 
The manuscript discusses and presents results for an experiment where laminated plate theory (LPT) in conjunction with mechanical testing was used to determine the correct layup orientations of a carbon-fiber composite laminate. The mechanical test setup consisted of a four-point bend test to induce a known moment in the mid-section of the composite specimen where strain gauges were attached to measure strains during flexure. A list of five possible layup configurations was given, in which, one of the five was known to be the correct layup configuration. Using LPT with the given loading conditions and ply orientations as inputs, predicted strains were calculated for each of the five possible layups. The correct layup was chosen by calculating L2 error between the calculated strains and experimental strains and choosing the layup that resulted in the smallest error.
\\ 
\\
\textit{\underline{Recommendation:} Accept with major revisions}

\section{Comments on the technical aspects of the manuscript}
The introduction shows that a thorough literature review was done prior to implementation of analysis. The introduction also provides a good context of the authors motivations as well as the contribution goals of the manuscript. 
 \\
 \\
More justification is needed for omitting strain gauge measurements of gauge 2 in the analysis.  It was claimed that the orientation of gauge 2 (``y-direction'') resulted in relatively small strain measurements and thus should be left out of analysis. It is unclear as to what the threshold is and how this threshold is justified. What is too small and why? Additionally, there is no mention of what LPT would predict for strains in the y-direction, which undermines the results.  It would be beneficial to  include LPT calculations for both directions and to note the observed differences.
\\
\\
In the ``Errors and Uncertainties" section it was noted that the specimen experienced a load exceeding 300 N. This is inconsistent with the loads presented prior and and should be mentioned so. Clarification is also needed as to wether this was intentional or not. The actual magnitude of this load should also be reported since. It should also be mentioned wether any signs of failure, such as cracking noises or permanent deformations, were observed during this event.
\\
\\
The term $\bar Q_{i,j}$ in equations 1b, 1c, and 1d is not discussed or defined with in the manuscript. Equation 3 seems to be unnecessary and might be better convayed as a conditional exception in a sentance. Equations 8a, and 8b have several issues. First, the variables $m_x$ and $P$ are not defined.  Secondly, there is clearly an error in equation 8b since $\frac{m_x}{42 mm} \neq \frac{1000m_x}{49}N$. Lastly, It should be mentioned that $M_x$ is the resulting moment normalized by the width of the specimen which is necessary for LPT calculations.
\\
\\
The calculated strains from LPT were transformed according to the miss-alignment of strain gauges. The motivation for transforming the calculated strains instead of measured strains was unclear and should be explained with more detail.  What is advantage of using this method as opposed to strain gauge rosette equations?
\\
\\
The manuscript contains a few inconsistencies. The first being, two different coordinate notations, $(x,y)$ and $(0,90)$, were used in describing  gauge locations on the specimen. Additionally, strains of all five laminate layups were mentioned to be calculated using LPT, but only three are presented in the results. Lastly, the notation, $N$, was used to indicate number of plies as well as normal forces.
\\
\\
The test method for ASTM D6272-17 is not being replicated in this experiment as claimed. For example the standard calls for a given displacement rate but the load was incremented instead. The mentioning of this standard should be left out of the manuscript.  
\section{Comments related to non-technical aspects of the manuscript} 
%The document over all was well written with the exception of a some grammatical errors, and document %structure.  
The manuscript is well written over all. However the document does have several grammatical errors, for example in line 224 ``... there is an overall uncertainty associated with the each of the measured strain values." (the word ``the" should be left out before the word``each"). Additionally some sentences would read better if broken up in to separate sentences, like the sentence starting with ``ImageJ ..." in line 199. 
Section 2 could benefit from some reorganization. It seems as if the there are too many sub-sections which can become confusing to the reader in regards to the context of the experiment.  Flow and clarity could be improved by separating methods and theory from procedures as well as presenting results separate from discussion.
\end{document}